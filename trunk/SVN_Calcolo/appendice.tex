\newpage
\appendix

\chapter{Appendice}
\section{Algoritmo DFT}
\verbatiminput{matlab/dft.m}
\section{Algoritmo IDFT}
\verbatiminput{matlab/idft.m}
\section{Algoritmo FFT/IFFT - DC - ITER}
Implementazione iterativa dell'algoritmo di Cooley-Tukey radix-2 con calcolo dei \emph{twiddle factor} attraverso \emph{direct call}
\verbatiminput{matlab/fft_dc_iter.m}
\section{Algoritmo FFT/IFFT - RM - ITER}
Implementazione iterativa dell'algoritmo di Cooley-Tukey radix-2 con calcolo dei \emph{twiddle factor} attraverso \emph{repeated multiplication}
\verbatiminput{matlab/fft_rm_iter.m}
\section{Algoritmo FFT/IFFT - VS - ITER}
Implementazione iterativa dell'algoritmo di Cooley-Tukey radix-2 con calcolo dei \emph{twiddle factor} attraverso \emph{subvector scaling}
\verbatiminput{matlab/fft_vs_iter.m}

\begin{comment}
\section{FFT/IFFT 1D float C\#}
\begin{verbatim}
    /// <summary>
    /// Metodo per il calcolo della fft su un vettore di float
    /// </summary>
    /// <param name="data">vettore di float a coppie successive (Re(a),Im(a))</param>
    /// <param name="length">lunghezza della fft </param>
    /// <param name="direction">fft o ifft?</param>
    
		public static void	FFT( float[] data, int length, FourierDirection direction ) {
		
			Fourier.SyncLookupTableLength( length );
			
			int ln = Fourier.Log2( length );

			
			Fourier.ReorderArray( data );

			// fft o ifft?
			int signIndex = ( direction == FourierDirection.Forward ) ? 0 : 1;
			
			// implementazione iterativa butterfly CT
			int N = 1;
			
			//Calcolo le matrici A_i per i=1,...,ln
			
			for( int level = 1; level <= ln; level ++ ) {
				int M = N;
				N <<= 1;
				
				//lookup dei tf da utilizzare 
				
				float[] uRLookup = _uRLookupF[ level, signIndex ];
				float[] uILookup = _uILookupF[ level, signIndex ];

				// Butterfly operation su gruppi di 2*M=N elementi
				 
				for( int j = 0; j < M; j ++ ) {
					float uR = uRLookup[j];
					float uI = uILookup[j];
				
					for( int evenT = j; evenT < length; evenT += N ) {
					  // posizione dell'elemento pari
						int even = evenT << 1;
						// posizione dell'elemento dispari
						int odd = ( evenT + M ) << 1;
						
						//ricombinazine
						
						float r = data[ odd ];
						float i = data[ odd+1 ];

						// moltiplicazione odd*tf
						float odduR = r * uR - i * uI;
						float odduI = r * uI + i * uR;

						r = data[ even ];
						i = data[ even+1 ];

						data[ even ]	= r + odduR;
						data[ even+1 ]	= i + odduI;

						data[ odd ]		= r - odduR;
						data[ odd+1 ]	= i - odduI;
					}
				}
			}
		}
\end{verbatim}
\end{comment}
\section{FFT 1-D C\#}
\begin{verbatim}
		/// <summary>
		/// FFT 1-D su ComplexF[]
		/// </summary>
		/// <param name="data">vettore di struct con float .Re .Im</param>
		/// <param name="length">lunghezza della fft </param>
		/// <param name="direction">fft o ifft?</param>
    
public static void	FFT( ComplexF[] data, int length, FourierDirection direction ) {

		Fourier.SyncLookupTableLength( length );
		int ln = Fourier.Log2( length );

		// riordina  il vettore di ingresso in bit reversal
		Fourier.ReorderArray( data );

		// fft o ifft?
		int signIndex = ( direction == FourierDirection.Forward ) ? 0 : 1;
		
		// implementazione iterativa Cooley-Tukey
	  int N = 1;
			
		//Calcoli associati alle matrici A_i per i=1,...,ln

		for( int level = 1; level <= ln; level ++ ) {
				int M = N;
				N <<= 1;

				//lookup dei tf da utilizzare dalla tabella precalcolata 
				
				float[] uRLookup = _uRLookupF[ level, signIndex ];
				float[] uILookup = _uILookupF[ level, signIndex ];

				// Butterfly operation su gruppi di 2*M=N elementi
				
				for( int j = 0; j < M; j ++ ) {
            float uR = uRLookup[j];
            float uI = uILookup[j];

            for( int even = j; even < length; even += N ) {
                int odd	 = even + M;

                float	r = data[ odd ].Re;
                float	i = data[ odd ].Im;

                float	odduR = r * uR - i * uI;
                float	odduI = r * uI + i * uR;

                r = data[ even ].Re;
                i = data[ even ].Im;

                data[ even ].Re	= r + odduR;
                data[ even ].Im	= i + odduI;

                data[ odd ].Re	= r - odduR;
                data[ odd ].Im	= i - odduI;
            }
        }
    }
}
\end{verbatim}

\section{FFT 2-D C\#}
\begin{verbatim}
/// <summary>
/// FFT 2-D su ComplexF[]
/// </summary>
/// <param name="data"></param>
/// <param name="xLength"></param>
/// <param name="yLength"></param>
/// <param name="direction"></param>

public static void	FFT2( ComplexF[] data, int xLength, int yLength,
                          FourierDirection direction ) {
    int xInc = 1;
    int yInc = xLength;
    //fft 1-D per colonne
    if( xLength > 1 ) {
        Fourier.SyncLookupTableLength( xLength );
        for( int y = 0; y < yLength; y ++ ) {
            int xStart = y * yInc;
            Fourier.LinearFFT_Quick( data, xStart, xInc, xLength, direction );
        }
    }
    //fft 1-D per righe
    if( yLength > 1 ) {
        Fourier.SyncLookupTableLength( yLength );
        for( int x = 0; x < xLength; x ++ ) {
            int yStart = x * xInc;
            Fourier.LinearFFT_Quick( data, yStart, yInc, yLength, direction );
        }
    }
}

/// <summary>
/// la matrice data � memorizzata su un array
/// permette di selezionare opportunamente gli elementi su una riga/colonna
/// metterli su un buffer e calcolarne la trasformata
/// </summary>
/// <param name="data">matrice memorizzata come vettore</param>
/// <param name="start">posizione primo elemento nel vettore</param>
/// <param name="inc">posizione elementi successivi nel vettore +inc</param>
/// <param name="length">lunghezza della fft da calcolare</param>
/// <param name="direction">fft/ifft</param>

private static void	LinearFFT_Quick( ComplexF[] data, int start, int inc,
                                    int length, FourierDirection direction ) {

    // copio nel buffer
    ComplexF[]	buffer = null;
    LockBufferCF( length, ref buffer );
    int j = start;
    for( int i = 0; i < length; i ++ ) {
        buffer[ i ] = data[ j ];
        j += inc;
    }
    //fft 1-D
    FFT( buffer, length, direction );
    // copio dal buffer
    j = start;
    for( int i = 0; i < length; i ++ ) {
        data[ j ] = buffer[ i ];
        j += inc;
    }
    UnlockBufferCF( ref buffer );
}
\end{verbatim}
