\begin{appendices}

\chapter[Appendix]{}

\section{Probability}
\textcolor{red}{sezione provvisoria}
\subsection{Chebyshev's inequality}
Let $X$ be a \emph{random variable} with expected value $\mu$ and finite variance $\sigma^{2}$. Then for any real number $k>0$,
\begin{equation}
\Pr(\left|X-\mu\right|\geq k)\leq\frac{\sigma^{2}}{k^2}
\label{eq:cheby}
\end{equation}
\subsection{Binomial Distribution}

$B(n,p)$
\subsubsection{Poisson Approximation}
The binomial distribution converges towards the Poisson distribution as the number of trials goes to infinity while the product $np$ remains fixed. Therefore the Poisson distribution with parameter $\lambda = np$ can be used as an approximation to $B(n, p)$ of the binomial distribution if $n$ is sufficiently large and $p$ is sufficiently small. According to two rules of thumb, this approximation is good if $n \geq 20$ and $p \leq 0.05$, or if $n \geq 100$ and $np \leq 10$.
\subsubsection{Normal Approximation}
If $n$ is large enough, then the skew of the distribution is not too great. In this case, if a suitable continuity correction is used, then an excellent approximation to $B(n, p)$ is given by the normal distribution $\mathcal{N}(np,np(1-p))$\\
The approximation generally improves as $n$ increases and is better when $p$ is not near to 0 or 1. Various rules of thumb may be used to decide whether $n$ is large enough, and $p$ is far enough from the extremes of zero or unity:
One rule is that both $n p$ and $n(1-p)$ must be greater than 5. However, the specific number varies from source to source, and depends on how good an approximation one wants; some sources give 10.\\

oppure dal libro $np(1-p)\geq 10$.




\subsection{Poisson Distribution}

\subsection{Normal Distribution}

 $\mathcal{N}(\mu,\sigma^{2})$
\begin{equation}
f(x)= \frac{1}{\sqrt{2\pi\sigma^{2}}}\exp{-\frac{(x-\mu)^{2}}{2\sigma^{2}}}
\end{equation}

\section{Greenberg bounded \emph{m}-moments}
\label{sec:greenberg-moments}
In general for base $b$ greenberg algorithm the first and second moments are bounded by:

\begin{equation}
\phi(b)= \frac{1}{\log b} \int_{0}^{\infty} \! e^{-x}(1+x) \prod_{k=1}^{\infty}(1-e^{-b^{k}x}(1+b^{k}x))x^{-2} \, dx
\label{eq:greenberg-b-phi}
\end{equation}

\begin{equation}
\Phi(b)= \frac{1}{\log b} \int_{0}^{\infty} \! e^{-x}(1+x) \prod_{k=1}^{\infty}(1-e^{-b^{k}x}(1+b^{k}x))x^{-3} \, dx
\label{eq:greenberg-b-Phi}
\end{equation}

\lstinputlisting{matlab/Greenberg_base_b/phi/calculatephi.m}
\lstinputlisting{matlab/Greenberg_base_b/phi/f4.m}
\lstinputlisting{matlab/Greenberg_base_b/phi/f3.m}
\lstinputlisting{matlab/Greenberg_base_b/phi/f2.m}
\textcolor{red}{note sul calcolo\\ va velocemente a 0 quind basta considerare un intervallo iniziale  limitato\\
anche per produttoria con k è lo stesso.\\ risolto con quad matlab}


\section{CBT Estimate Experimental Distribution}

Following tables \ref{CBT-table-1} shows the behavior of CBT Algorithm (section \ref{cbt-estimation}) for estimation.  Simulation was implemented in matlab.
The resulting distribution of $\hat{n}$ fixed $n$ is the result of averaging \numprint{100000} runs of CBT Algorihm applied on uniformly random generated nodes ID batches.\\ 

\begin{table}[H]
\caption[Experimentally computed CBT Estimate Distributon]{Experimentally computed CBT Estimate Distributon. Table 1/3}
\label{CBT-table-1}
\resizebox{\textwidth}{!}{
\begin{tabular}{r|cccccccccccc}
n&$\hat{n}:$&2&4&8&16&32&64&128&256&512&1024&2048 \\\hline

2 &&0.499 &0.253 &0.125 &0.061 &0.031 &0.015 &0.007 &0.004 &0.002 &9e-04 &4e-04\\\hline

4 &&&0.189 &0.303 &0.225 &0.133 &0.072 &0.038 &0.020 &0.010 &0.005 &0.002\\\hline

8 &&&0.055 &0.212 &0.261 &0.201 &0.126 &0.071 &0.037 &0.019 &0.009 &0.005\\\hline

16 &&&8e-04 &0.070 &0.209 &0.252 &0.197 &0.125 &0.070 &0.038 &0.019 &0.010\\\hline

32 &&&&0.003 &0.075 &0.213 &0.249 &0.195 &0.123 &0.069 &0.037 &0.018\\\hline

64 &&&&&0.004 &0.077 &0.208 &0.250 &0.193 &0.124 &0.069 &0.037\\\hline

128 &&&&&&0.005 &0.081 &0.208 &0.247 &0.191 &0.123 &0.069\\\hline

256 &&&&&&2e-05 &0.006 &0.079 &0.209 &0.246 &0.193 &0.123\\\hline

512 &&&&&&&&0.005 &0.081 &0.207 &0.245 &0.193\\\hline

1024 &&&&&&&&&0.005 &0.080 &0.208 &0.245\\\hline

2048 &&&&&&&&&2e-05 &0.005 &0.080 &0.209\\\hline

4096 &&&&&&&&&&1e-05 &0.006 &0.082\\\hline

8192 &&&&&&&&&&&1e-05 &0.006\\\hline

16384 &&&&&&&&&&&&2e-05\\\hline

32768 &&&&&&&&&&&&\\\hline

\end{tabular}
}

\end{table}
\begin{table}[H]
\ContinuedFloat
\caption[]{Experimentally computed CBT Estimate Distributon. Table 2/3}
\resizebox{\textwidth}{!}{
\begin{tabular}{r|cccccccccccc}
n&$\hat{n}:$&4096&8192&16384&32768&$2^{16}$&$2^{17}$&$2^{18}$&$2^{19}$&$2^{20}$&$2^{21}$&$2^{22}$ \\\hline

2 &&2e-04 &1e-04 &1e-04 &&&&1e-05 &&&&\\\hline

4 &&0.001 &5e-04 &3e-04 &1e-04 &6e-05 &4e-05 &&2e-05 &1e-05 &&\\\hline

8 &&0.002 &0.001 &6e-04 &3e-04 &9e-05 &8e-05 &4e-05 &1e-05 &&&\\\hline

16 &&0.005 &0.003 &0.001 &6e-04 &3e-04 &2e-04 &6e-05 &2e-05 &2e-05 &&\\\hline

32 &&0.009 &0.005 &0.003 &0.001 &6e-04 &3e-04 &1e-04 &1e-04 &6e-05 &1e-05 &1e-05\\\hline

64 &&0.019 &0.009 &0.005 &0.002 &0.001 &7e-04 &3e-04 &7e-05 &4e-05 &&2e-05\\\hline

128 &&0.037 &0.019 &0.010 &0.005 &0.003 &0.001 &5e-04 &4e-04 &2e-04 &5e-05 &4e-05\\\hline

256 &&0.068 &0.038 &0.019 &0.009 &0.005 &0.002 &0.001 &6e-04 &3e-04 &6e-05 &8e-05\\\hline

512 &&0.123 &0.071 &0.037 &0.019 &0.010 &0.005 &0.002 &0.001 &6e-04 &3e-04 &2e-04\\\hline

1024 &&0.193 &0.122 &0.070 &0.037 &0.019 &0.010 &0.005 &0.002 &0.001 &6e-04 &2e-04\\\hline

2048 &&0.246 &0.194 &0.123 &0.068 &0.037 &0.019 &0.010 &0.005 &0.002 &0.001 &6e-04\\\hline

4096 &&0.210 &0.246 &0.193 &0.121 &0.068 &0.037 &0.019 &0.009 &0.004 &0.003 &0.001\\\hline

8192 &&0.080 &0.208 &0.247 &0.192 &0.123 &0.070 &0.037 &0.019 &0.009 &0.005 &0.002\\\hline

16384 &&0.006 &0.080 &0.208 &0.247 &0.192 &0.123 &0.069 &0.037 &0.019 &0.010 &0.005\\\hline

32768 &&&0.006 &0.079 &0.209 &0.248 &0.194 &0.122 &0.069 &0.036 &0.019 &0.010\\\hline

\end{tabular}
}

\end{table}
\begin{table}[H]\ContinuedFloat
\caption[]{Experimentally computed CBT Estimate Distributon. Table 3/3}
\resizebox{\textwidth}{!}{
\begin{tabular}{r|ccccccccccc}
n&$\hat{n}:$&$2^{23}$&$2^{24}$&$2^{25}$&$2^{26}$&$2^{27}$&$2^{28}$&$2^{29}$&$2^{30}$&$2^{31}$&$2^{32}$ \\\hline

2 &&&&&&&&&&&\\\hline

4 &&&&&&&&&&&\\\hline

8 &&&&&&&&&&&\\\hline

16 &&&&&&&&&&&\\\hline

32 &&1e-05 &&&&&&&&&\\\hline

64 &&2e-05 &&&&1e-05 &&&&&\\\hline

128 &&4e-05 &&1e-05 &&&&&&&\\\hline

256 &&4e-05 &1e-05 &&2e-05 &&&1e-05 &&&\\\hline

512 &&7e-05 &2e-05 &1e-05 &3e-05 &1e-05 &&&&&\\\hline

1024 &&2e-04 &1e-04 &4e-05 &&2e-05 &1e-05 &&&&\\\hline

2048 &&3e-04 &1e-04 &3e-05 &3e-05 &3e-05 &3e-05 &&&&\\\hline

4096 &&6e-04 &3e-04 &9e-05 &7e-05 &1e-05 &&&&&\\\hline

8192 &&0.001 &8e-04 &3e-04 &2e-04 &8e-05 &7e-05 &2e-05 &3e-05 &1e-05 &\\\hline

16384 &&0.002 &0.001 &6e-04 &3e-04 &1e-04 &1e-04 &4e-05 &&&\\\hline

32768 &&0.005 &0.002 &0.001 &6e-04 &3e-04 &2e-04 &1e-04 &1e-05 &2e-05 &1e-05\\\hline

\end{tabular}
}
\end{table}

\lstinputlisting{matlab/CBT/cbtsimpletest.m}
\lstinputlisting{matlab/CBT/cbtsplit.m}
\lstinputlisting{matlab/CBT/cbtfulltest.m}
%\verbatiminput{matlab/CBT/cbtfulltest.m}

\section{Greenberg Estimate Distribution}
In following table \ref{basic-greenberg-stop-probabilities} we report how the end  up probability (equation \ref{eq:bgstopprobability}) is distributed among slots given a batch of size $n$.  Column ``$n$'' lists  the considered batch sizes. $\hat{n}$ is the resulting estimation (without corrections) when ending up in the underneath slot.\\  For sake of simplicity considered values are all powers of 2.\\
Datas presented were post-processed to become more accessible:
\begin{itemize}
\item values above $10^{-3}$ are reported in format ('\emph{\%1.3f}');
\item values below $10^{-12}$ are not presented since are tight close to 0.
\item other values are presented in exponential notation and rounded to the first meaningful digit ('\emph{\%1.e}')
\end{itemize}


\begin{sidewaystable}
%%%%%
\flushleft
\resizebox{25cm}{!}{
\begin{tabular}{r|ccccccccccccccccccccccc}
&$\hat{n}$& 2 &4 &8 &16 &32 &64 &128 &256 &512 &1024 &2048 &4096 &8192 &16384 &32768 &65536\\
n & slot:& 1 &2 &3 &4 &5 &6 &7 &8 &9 &10 &11 &12 &13 &14 &15 & 16\\ 
\toprule
1 &&1.000 &&&&&&&&&\\\hline

2 &&0.750 &0.234 &0.015 &2e-04 &1e-06 &9e-10 &&&&&&&&\\\hline

4 &&0.312 &0.508 &0.166 &0.014 &3e-04 &2e-06 &2e-09 &&&&&&&\\\hline

8 &&0.035 &0.354 &0.450 &0.147 &0.013 &3e-04 &2e-06 &4e-09 &1e-12 &&&&&&\\\hline

16 &&3e-04 &0.063 &0.363 &0.422 &0.138 &0.013 &3e-04 &2e-06 &4e-09 &2e-12 &&&&&\\\hline

32 &&8e-09 &0.001 &0.078 &0.366 &0.409 &0.134 &0.013 &3e-04 &2e-06 &4e-09 &2e-12 &&&&\\\hline

64 &&&2e-07 &0.002 &0.084 &0.367 &0.402 &0.131 &0.013 &3e-04 &2e-06 &4e-09 &2e-12 &&&\\\hline

128 &&&&7e-07 &0.002 &0.088 &0.367 &0.399 &0.130 &0.013 &3e-04 &2e-06 &5e-09 &2e-12 &&\\\hline

256 &&&&&1e-06 &0.003 &0.090 &0.368 &0.397 &0.130 &0.013 &3e-04 &2e-06 &5e-09 &2e-12 & \\\hline

512 &&&&&&2e-06 &0.003 &0.090 &0.368 &0.397 &0.130 &0.012 &3e-04 &2e-06 &5e-09 &2e-12 \\\hline
1024 &&&&&&&2e-06 &0.003 &0.091 &0.368 &0.396 &0.129 &0.012 &3e-04 &2e-06 &5e-09 &2e-12 \\
\bottomrule
\end{tabular}
}

\begin{tabular}{c}
\\
\end{tabular}
\resizebox{25cm}{!}{
\begin{tabular}{r|ccccccccccccccccccccccc}
&$\hat{n}$&128 &256 &512 &1024 &2048 &4096 &8192 &16384 &32768 &65536 &$2^{17}$ &$2^{18}$ &$2^{19}$ &$2^{20}$ &$2^{21}$ &$2^{22}$ \\ 
n & slot:& 7 &8 &9 &10 &11 &12 &13 &14 &15 &16 &17 &18 &19 &20 &21 &22 \\ 
\toprule


2048 &&2e-06 &0.003 &0.091 &0.368 &0.396 &0.129 &0.012 &3e-04 &2e-06 &5e-09 &2e-12 &&&&&\\\hline

4096 &&&2e-06 &0.003 &0.091 &0.368 &0.396 &0.129 &0.012 &3e-04 &2e-06 &5e-09 &2e-12 &&&&\\\hline

8192 &&&&2e-06 &0.003 &0.091 &0.368 &0.396 &0.129 &0.012 &3e-04 &2e-06 &5e-09 &2e-12 &&&\\\hline

16384 &&&&&2e-06 &0.003 &0.091 &0.368 &0.396 &0.129 &0.012 &3e-04 &2e-06 &5e-09 &2e-12 &&\\\hline

32768 &&&&&&2e-06 &0.003 &0.091 &0.368 &0.396 &0.129 &0.012 &3e-04 &2e-06 &5e-09 &2e-12 &\\\hline

65536 &&&&&&&2e-06 &0.003 &0.091 &0.368 &0.396 &0.129 &0.012 &3e-04 &2e-06 &5e-09 &2e-12\\

\bottomrule
\end{tabular}
}
%%%%
\caption{Analytically computed basic Greeenberg Estimate Distribution}
\label{basic-greenberg-stop-probabilities}

\end{sidewaystable}

\end{appendices}
